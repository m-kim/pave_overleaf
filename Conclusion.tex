\section{Conclusions}

We have presented an in situ in-transit framework to couple scientific visualization applications to machine learning applications. Our work offers a distributable and scalable implementation and framework to allow researchers and practitioners to easily integrate state of the art deep learning tools for scientific visualization. The ease of itegration is afforded by the approachable PyTorch and robust visualisation resource VTK-m for in situ rendering or scientific simulation for HPC systems. We presented a case study utilising cGANs for generative visualisation which also offers the prospect of quickly and accurately visualise conditionally dependent data such as light path global illumination's dependence on scene geometry. Subsequently we here then also offer a framework for visualisation within C++ with the use of VTK-m as well as Python in tandem or independently. 

In the future, we would like to scale beyond two GPUs to examine the trade offs in performance between the visualization and machine learning tasks. Further, we would like to integrate other visualization tools, such as Intel's OSPRay or Nvidia's OptiX, as well as, other machine learning frameworks such as TensorFlow or Caffe to examine trade-offs in performance and quality as well. Finally, we would like to validate whether machine learning filters for global illumination are of sufficient fidelity to use for scientific visualization applications in HPC.