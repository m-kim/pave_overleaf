\documentclass[conference]{IEEEtran}
\IEEEoverridecommandlockouts
% The preceding line is only needed to identify funding in the first footnote. If that is unneeded, please comment it out.
\usepackage{minted}
\usepackage{cite}
\usepackage{amsmath,amssymb,amsfonts}
\usepackage{algorithmic}
\usepackage{graphicx}
\usepackage{textcomp}
\usepackage{xcolor}
\usepackage[labelformat=simple]{caption}

%https://tex.stackexchange.com/questions/113459/ieeetran-caption-question
%IEEETran doesn't like subcaption
\makeatletter
\let\MYcaption\@makecaption
\makeatother

\usepackage{subcaption}

\makeatletter
\let\@makecaption\MYcaption
\makeatother

\def\BibTeX{{\rm B\kern-.05em{\sc i\kern-.025em b}\kern-.08em
    T\kern-.1667em\lower.7ex\hbox{E}\kern-.125emX}}
\begin{document}

\title{PAVE: An In Situ Framework for Scientific Visualization and Machine Learning Coupling\\
}

\author{\IEEEauthorblockN{1\textsuperscript{st} Samuel Leventhal}
\IEEEauthorblockA{\textit{University of Utah School of Computing} \\
\textit{Scientific Computing and Imaging Institute}\\
Salt Lake City, UT., USA \\
samlev@cs.utah.edu}
\and
\IEEEauthorblockN{2\textsuperscript{nd} Mark Kim}
\IEEEauthorblockA{\textit{Oak Ridge National Laboratory} \\
Oak Ridge TN., USA \\
kimmb@ornl.gov}
\and
\IEEEauthorblockN{3\textsuperscript{rd} David Pugmire}
\IEEEauthorblockA{\textit{Oak Ridge National Laboratory} \\
Oak Ridge TN., USA \\
pugmire@ornl.gov}
}

\maketitle

\begin{abstract}
    Machine learning (ML) has emerged as a tool for understanding data at scale. However, this new methodology comes at a cost because ML requires the use of even more HPC resources to generate ML algorithms. In addition to the compute resources required to develop ML algorithms, ML does not side step one of the biggest challenges on leading-edge HPC systems: the increasing gap between compute performance and I/O bandwidth. This has led to a strong push towards \textit{in situ}, processing the data as it is generated, strategies to mitigate the I/O bottleneck. Unfortunately, there are no \textit{in situ} frameworks dedicated to coupling scientific visualization and ML at scale to develop ML algorithms for scientific visualization.
    
    To address the ML and \textit{in situ} visualization gap, we introduce PAVE. PAVE is an \textit{in situ} framework which addresses the data management needs between visualisation and machine learning tasks. We demonstrate our framework with a case study that accelerates physically-based light rendering, path-tracing, through the use of a conditional Generative Adversarial neural Network (cGAN). PAVE couples the training over path-traced images resulting in a generative model able to produce scene renderings with accurate light transport and global illumination of a quality comparable to offline approaches in a more efficient manner.
    
\end{abstract}

\begin{IEEEkeywords}
    VTKm, neural networks, generative adversarial network, in situ, PyTorch, path-tracing
\end{IEEEkeywords}

\newcommand{\markcomment}[1]{\textcolor{red}{#1}}

% \begin{teaserfigure}
%     \includegraphics[width=\textwidth]{buffer_results_teaser.png}
%     \caption{\textmd{Rendered Conditional Geometry Buffers ({\bf left set}) and artificial rendering with conditional generative adversarial neural network ({\bf right couple}) comparing ground truth path traced rendering ({\bf left}) with image generated ({\bf right}).}}
%     \Description{Conditional Buffers and path traced rendered with VTKm to be used for training a PyTorch conditional generative adversarial network.}
%     \label{teaser}
%   \end{teaserfigure}
\section{Introduction}
Machine learning (ML) has become a significant driving force within the HPC community. New ML algorithms are being developed to analyze the data deluge coming from simulation codes, ``Big Data'' problems, as well as scientific instruments. Unfortunately, I/O bandwidth continues to lag behind compute~\cite{7426274}. However, \textit{in situ} processing~\cite{doi:10.1111/cgf.12930}, which processes the data where it is generated, have stepped in to cope with this I/O bottleneck. These frameworks have become more common in the modern HPC environment. 

Machine learning has a cost, though. Traditionally a researcher develops an algorithm to combine with data to produce a scientific visualization output such as a rendered image or a triangular mesh (Fig.~\ref{fig:ml-vs-trad}). Developing ML algorithms is the inverse of the traditional programming model: data is processed with traditional output (simulation data, high-quality rendered images, etc) to generate ML algorithms. Significant amount of computation as well as copious amounts of traditional output can be required to construct sophisticated ML algorithms. To perform this coupling efficiently at scale, an \textit{in situ} processing is required.

\begin{figure}
    \includegraphics[width=\linewidth]{ML-data-output-program}
    \caption{Traditional programming vs machine learning.}
    \label{fig:ml-vs-trad}
  \end{figure}

In this work we present PAVE, an \textit{in situ} framework for coupling scientific visualization and machine learning. The purpose of this framework is to offer a solution which provides the means necessary to couple machine learning implementations at scale and \textit{in situ} to aid or be the basis of scientific visualisation tasks. We accomplish this by providing read, write and in place data transfer functionality between learning models and visualisation task implementations. To our knowledge, this is the first scientific visualization-machine learning \textit{in situ} framework.

Further, we present a case study of coupling a path-tracer, a physically-based light rendering model, with a neural network to build a filter for a more efficient, but accurate light transport. The case study utilizes VTK-m, a toolkit for massively threaded architectures for scientific visualization and Python, an increasingly popular language within the machine learning community due to robust libraries available for neural networks such as PyTorch. The resulting work accomplishes this combination by utilising VTK-m to construct a path trace rendering tool able to fluidly and efficiently communicate to a cGAN by means of PAVE during training.   The resulting generative model serves as a real-time filter for rendering globally illuminated images which accurately approximate diffuse indirect illumination and soft shadows with quality comparable to offline approaches. 

\begin{figure*}
    \includegraphics[width=\linewidth]{Teaser.png}
    \caption{Rendered Conditional Geometry Buffers ({\bf left set}) and artificial rendering with conditional generative adversarial neural network ({\bf right couple}) comparing ground truth path traced rendering ({\bf left}) with image generated ({\bf right}).}
    \label{teaser}
  \end{figure*}


\input{RelatedWorks.tex}
\section{Coupling Scientific Visualization and Machine Learning}
PAVE is an \textit{in situ} framework for coupling scientific visualization and machine learning. The resulting framework consists of two core components, visualization output coupled to a machine learning application. A scientific visualization application is coupled through the C++ interface, and the output is sent through PAVE to the machine learning application to construct a new algorithm (Fig.~\ref{fig:ml-vs-trad}. PAVE then allows each task to communicate results among each other seamlessly. The scientific visualisation task developed by the user can then provide resulting visualisations to a separately developed learning model as input or employ a learning model within the visualisation task. 
\subsection{User Provided Visualisation}
\label{sec:PAVEvis}
PAVE couples visualization and machine learning through an \textit{in situ} framework that supports \textit{in situ}, in transit, and post-processing. It provides a unified interface which allows, among other things, memory-to-memory transport between applications during the I/O phase of a simulation, i.e. in transit, With the  unified interface, applications can be coupled either memory-to-memory or through the filesystem. The results for the scalable, \textit{in situ} scientific visualization task designed by the user, are passed to a learning model through a single interface and the visualization can remain fully scalable to distributed systems because of PAVE. For this same reason in the provided example in section \ref{ex} we chose VTK-m arrays as the data structures of the visualisation task. 

\setminted{fontsize=\scriptsize,baselinestretch=1} 
\begin{listing}[htb]
\noindent\rule{0.5\textwidth}{1pt}
\inputminted{cpp}{pave_pt.py}
%\inputminted{python}{adiosdataloader.py}
\noindent\rule{0.5\textwidth}{1pt}
\caption{C++ Interface for PAVE}
\label{fig:cpp_interface}
\end{listing}

The interface in Listing~\ref{fig:cpp_interface} demonstrates the visualisation component of PAVE. PAVE is initialized with a name for the training set that will be generated by the visualization tool, that will be used with the machine learning application. As each datum (image, text, etc.) is generated, it is ``saved'' to the training set. However, PAVE will buffer the data until either an explicit ``flush'' is called, which will flush the current in-memory data to disk, or the training set has been completed. 

\subsection{User Defined Machine Learning Application}
\label{sec:PAVElearn}
PAVE allows researchers or practitioners to implement their learning algorithms in the increasingly popular language Python due to having a robust library for learning tasks and notably neural networks. 

\begin{listing}[htb]
\noindent\rule{0.5\textwidth}{1pt}
\inputminted{python}{pave.py}
%\inputminted{python}{adiosdataloader.py}
\noindent\rule{0.5\textwidth}{1pt}
\caption{Python Interface}
\label{fig:python_interface}
\end{listing}

In Listing~\ref{fig:python_interface}, we demonstrate employing PAVE while training a PyTorch model. The training method for Model is able to request visualisation samples used during training based on some parameter used in the visualisation task or retrieve precomputed visualisation data by calling PAVE.
 
\subsection{Communication of Visualisation Data and Learning}

As demonstrated in Section~\ref{sec:PAVEvis} and PAVE allows for the user to save or pass data produced by the simulation and similarly the user would also be able to request results from the learning model depending on the application. Section \ref{sec:PAVElearn} demonstrates PAVE's support in requesting data in place from the user's visualization implementation during training of a predictive model used as example. 

\input{CaseStudy1.tex}




\section{Cornell Box Experiment}

To evaluate the quality of \textit{in situ} deep learning aided visualisations train the cGAN networks on rendered images of a Cornell box, a commonly used 3D modelling framework for quality assessment. We train the model using renderings of the Cornell box with high light sample count and depth computation per ray for various camera angle perspectives into the box along with the associated image geometry buffers for a given camera orientation. We then assess the quality of the models final generated renderings looking at the accuracy of global illumination. We then also demonstrate the performance of the models ability to render global illumination when given image buffers for a novel scene not used for training similar in content but not exact. The scene used for training is comprised of the classic set up with one overhead light source in the center of a white ceiling, a white back wall and a white floor. The remaining walls are then colored red on the left and green on the right in order to afford different colored light transport and demonstrate diffuse interreflection. The contents of the Cornell box are three cuboids of various shapes and sizes to provide diverse shading and diffused lighting. 

%\vspace{-1.5em}
\begin{figure}
\centering
\begin{subfigure}{0.25\textwidth}
\centering
\includegraphics[width=.9\textwidth]{sc-1080-d-45.png}
\caption{Path traced global illumination rendered}
\end{subfigure}%
\begin{subfigure}{.25\textwidth}
\centering
\includegraphics[width=.9\textwidth]{conditionals.png}
\caption{Geometry buffers}
\label{Gbuf}
\end{subfigure}
\caption{ (a) \textmd{Ground truth image rendered with VTK-m used for training.} (b) \textmd{Global illumination geometry buffers used as conditional variables for generative model. {\bf Top:} left, Albedo. right, Direct Lighting. {\bf Bottom:} left, Normals. right, Depth.}}
\end{figure}
%\vspace{-1em}

The conditional differed shading geometry buffers used are direct lighting, normal planes, depth and albedo as shown in figure \ref{teaser}.

\begin{figure}


\end{figure}

The geometry buffers serve as joint variables for the conditional probability distribution which the global illumination path traced images are considered to exist. The conditional arguments in this experiment then aid the cGAN in learning behavior of light paths given the geometry of a scene in question. 

\section{Results}
\subsection{Cornell Box Experiment Results}
The resulting generated images show promising results for deep learning aided in situ scientific visualisation. We observe the network successfully learned to emulate light transport in a realistic fashion with offline performance as can be seen in the highlighted region of figure \ref{fig:lighttransport}.

\begin{figure}
    \includegraphics[width=0.55\textwidth]{genDemo.png}
    \caption{Ground truth path traced image \textbf{(left)} and image generated from noise and conditional buffers by cGAN \textbf{(right)}.}
    \label{fig:lighttransport}
\end{figure}

What is more, though designed in a ``one network for one scene'' setting, the generative net proved to be adaptive and able to generate accurate renderings for not only unobserved camera orientation renderings during training but also varied scenes of a similar flavor when provided the conditional geometry buffers of the novel scene. 
%\vspace{-1.5em}
\begin{figure}
\centering
\begin{subfigure}{.5\textwidth}
\centering
\includegraphics[width=.9\textwidth]{discrimloss.png}
\end{subfigure}
\begin{subfigure}{.5\textwidth}
\centering
\includegraphics[width=.9\textwidth]{genloss.png}
\end{subfigure}
\caption{Generator loss converging to 0 and discriminator loss converging to 0.5 corresponding to inability to differentiate real from generated images.}
\end{figure}


\subsection{Solution Design Assessment}


3080 256x256 high quality ground truth images were rendered utilizing the VTK-m path tracer and two Nvidia RTX-2080 Ti GPUs requiring 12 hours. To generate 3080 images, a camera was positioned in a hemisphere around the Cornell Box, and moved along the hemisphere as the camera consistent ``looked'' at the center of the Cornell Box. This was the ground truth training set for the cGAN. Each image was considered a single datum in the path traced training set.

%17471.48user 9708.75system 4:26:31elapsed 169%CPU (0avgtext+0avgdata 13497260maxresident)

Training the cGAN on this image data set over 400 epochs using one GPU on the same machine took 9709 seconds. %Once trained the run time of applying the generative U-Net provided the conditional buffer set averages \_\_\_ seconds. 




\section{Conclusions}

We have presented an in situ in-transit framework to couple scientific visualization applications to machine learning applications. Our work offers a distributable and scalable implementation and framework to allow researchers and practitioners to easily integrate state of the art deep learning tools for scientific visualization. The ease of itegration is afforded by the approachable PyTorch and robust visualisation resource VTK-m for in situ rendering or scientific simulation for HPC systems. We presented a case study utilising cGANs for generative visualisation which also offers the prospect of quickly and accurately visualise conditionally dependent data such as light path global illumination's dependence on scene geometry. Subsequently we here then also offer a framework for visualisation within C++ with the use of VTK-m as well as Python in tandem or independently. 

In the future, we would like to scale beyond two GPUs to examine the trade offs in performance between the visualization and machine learning tasks. Further, we would like to integrate other visualization tools, such as Intel's OSPRay or Nvidia's OptiX, as well as, other machine learning frameworks such as TensorFlow or Caffe to examine trade-offs in performance and quality as well. Finally, we would like to validate whether machine learning filters for global illumination are of sufficient fidelity to use for scientific visualization applications in HPC.
%%
%% The acknowledgments section is defined using the "acks" environment
%% (and NOT an unnumbered section). This ensures the proper
%% identification of the section in the article metadata, and the
%% consistent spelling of the heading.
\section*{Acknowledgment}
 This research was supported in part by an appointment to the Oak Ridge National Laboratory ASTRO Program, sponsored by the U.S. Department of Energy and administered by the Oak Ridge Institute for Science and Education.



\bibliographystyle{IEEEtran}
\bibliography{pave_ref}
\vspace{12pt}

\end{document}
